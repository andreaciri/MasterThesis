%
% ------------------------------------------------------------------------ %
% 	ESTRATTI DA ''L'ARTE DI SCRIVERE CON LATEX''
%	DI LORENZO PANTIERI E TOMMASO GORDINI
% ------------------------------------------------------------------------ %
%
%
% ------------------------------------------------------------------------ %
% 	BIBLIOGRAFIA
% ------------------------------------------------------------------------ %
%
\cite{eco:tesi}
\cite[5]{eco:tesi}
\cite[5-9]{eco:tesi}
\cite[vedi][]{eco:tesi}
\cite[vedi][5]{eco:tesi}
%
Oltre ai due comandi appena visti, biblatex ne definisce altri per citazioni ancora diverse:
\textcite se la citazione è parte integrante del discorso: Eco [1977];
\parencite racchiude la citazione fra parentesi: [Eco, 1977];
\footcite mette la citazione in una nota, come qui1;
\supercite (solo per schemi numerici) mette la citazione in apice;
\fullcite riporta nella citazione l’intero riferimento bibliografico nello stile impostato: Umberto Eco 1977 Come si fa una tesi di laurea. Le materie umanistiche, Bompiani, Milano.
%
Per citare singole parti di un riferimento ci sono \citeauthor e \citeyear:
%
%
% ------------------------------------------------------------------------ %
% 	RIMPICCIOLIRE TABELLE
% ------------------------------------------------------------------------ %
%
\resizebox{0.95\textwidth}{!}{
\begin{tabular}
...
\end{tabular}}
%
% ------------------------------------------------------------------------ %
% 	RIFERIMENTI
% ------------------------------------------------------------------------ %
%
si veda la fig.~\vref{f5}
%
Per quanto riguarda invece il riferimento ad equazioni, è consigliabile
utilizzare il comando \eqref{...} al posto di \ref{...}
%
% ------------------------------------------------------------------------ %
% 	HYPERREF
% ------------------------------------------------------------------------ %
%
Si noti che se hyperref è caricato, subito prima di \addcontentsline va dato anche \phantomsection per evitare possibili errori nei collegamenti ipertestuali e nei segnalibri del documento finito
% ------------------------------------------------------------------------ %
Si possono avere tutti i collegamenti in nero e senza riquadri scrivendo semplicemente
\hypersetup{hidelinks}
% ------------------------------------------------------------------------ %
\newcommand{\mail}[1]{\href{mailto:#1}{\texttt{#1}}}
%
% ------------------------------------------------------------------------ %
% 	TITOLI LUNGHI
% ------------------------------------------------------------------------ %
%
Se, però, un titolo è troppo lungo per starci agevolmente (si noti che un titolo non dovrebbe mai andare a capo) o si hanno particolari esigenze, lo si può sostituire con un titolo alternativo più breve, da inserire nell’argomento facoltativo degli stessi comandi:
Si noti che il titolo breve comparirà anche nelle testatine, se previste dalla classe di documento in uso e che, ovviamente, non si può usare se il comando è asteriscato.
%
% ------------------------------------------------------------------------ %
% 	[...]
% ------------------------------------------------------------------------ %
%
\omissis
%
% ------------------------------------------------------------------------ %
% 	MATEMATICA
% ------------------------------------------------------------------------ %
%
L’unico modo corretto per scrivere queste formule è usare uno dei due ambienti matematici seguenti:
equation per le formule numerate;
equation* (di solito abbreviato in \[. . .\]) per quelle non numerate.
% ------------------------------------------------------------------------ %
NON DEVONO ESSERE PIÙ USATI:
-I dollari doppi $$. . .$$, che potrebbero compromettere la corretta spaziatura verticale delle formule o il funzionamento dell’opzione di classe fleqn [Fairbairns, 2012].
-Gli ambienti standard eqnarray e eqnarray* (per sistemi di formule numerate e non numerate rispettivamente), perché prima e dopo = inseriscono più spazio del dovuto
% ------------------------------------------------------------------------ %
Per inserire in una formula in display un (breve) testo in tondo e spaziato normalmente si usa il comando \text, esplicitando la spaziatura prima e dopo
% ------------------------------------------------------------------------ %
Le derivate si scrivono con il carattere ’, che produce il segno di primo.
% ------------------------------------------------------------------------ %
lo spazio sottile \, allontana dx da f(x).
% ------------------------------------------------------------------------ %
\overline{AB}
\overrightarrow{AB}
% ------------------------------------------------------------------------ %
Frecce. pag.73
% ------------------------------------------------------------------------ %
graffe anche in modo matematico devono essere precedute da \ 
% ------------------------------------------------------------------------ %
determinare le dimensioni dei delimitatori anche automaticamente, premettendo \left a quello di apertura e \right al corrispondente delimitatore di chiusura (se si vuole aprire l’espressione e non chiuderla, si userà \left. con il punto finale, e \right. in caso contrario).
% ------------------------------------------------------------------------ %
I vettori si scrivono di solito in tondo nero (corsivo, secondo le norme iso-uni) oppure in semplice corsivo matematico; talvolta, soprattutto nei testi di fisica, sono sormontati da una freccia. Nel primo caso si può usare il comando \mathbf; nel secondo il comando \bm del pacchetto bm; nel terzo il comando \vec. Può essere conveniente ridefinire nel preambolo quest’ultimo comando (si veda il paragrafo 11.1 a pagina 173):
\renewcommand{\vec}{\bm}
oppure
\renewcommand{\vec}{\mathbf}
% ------------------------------------------------------------------------ %
L’ambiente subequations produce sottonumerazioni
% ------------------------------------------------------------------------ %
spezzare e raggruppare formule pag.81
% ------------------------------------------------------------------------ %
Per evidenziare un’intera formula con uno sfondo colorato o con una cornice è utile il pacchetto empheq
%
% ------------------------------------------------------------------------ %
% 	SI
% ------------------------------------------------------------------------ %
%
\SI{23,4}{kg.m.s^{-2}} \\
$r=\SI{0,8768(11)e-15}{m}$ \\
\si{\joule\per\mole\per\kelvin}\\
\si{j.mol^{-1}.K^{-1}}
\SI{100}{\celsius} \\
\ang{1;2;3}
% ------------------------------------------------------------------------ %
Spaziare le cifre
La corretta scrittura dei numeri di cinque o più cifre prevede uno spazio sottile ogni tre cifre a partire da destra (come in 1 500 000). Per ottenerlo ci sono diverse possibilità: se i numeri da scrivere nel documento non sono molti, si può inserire a mano lo spazio sottile \, ; in caso contrario, risolve il problema il comando \num visto nel paragrafo precedente
%
% ------------------------------------------------------------------------ %
% 	TABELLE
% ------------------------------------------------------------------------ %
%
l’ambiente standard tabular, per tabelle che contengono prevalentemente testo;
l’ambiente standard array, per tabelle che contengono prevalentemente matematica;
il comando \includegraphics definito dal pacchetto graphicx, per includere nel documento le figure quando sono file esterni (come tutte quelle di questa guida).
% ------------------------------------------------------------------------ %
La tabella
\begin{center}
\begin{tabular}{ll}
\toprule
Alcaloide & Origine \\
\midrule
atropina  & belladonna \\
morfina   & papavero \\
nicotina  & tabacco \\
\bottomrule
\end{tabular}
\end{center}
mostra l’origine veg
% ------------------------------------------------------------------------ %
La tabella
\[
\begin{array}{ll}
\toprule
f(x)   & f’(x) \\
\midrule
x^n    & nx^{n-1} \\
e^x    & e^x \\
\sin x & \cos x \\
\bottomrule
\end{array}
\]
mostra le deriv
% ------------------------------------------------------------------------ %
i ambienti table e figure
Per rendere mobile un oggetto basta inserirne il relativo codice nell’ambiente standard table

\begin{table}[preferenze di collocazione]
...
\end{table}

se è una tabella, oppure in quello figure
se è una figura
% ------------------------------------------------------------------------ %
\caption[didascalia breve]{didascalia normale}

Il comando \label, da dare sempre dopo il corrispondente \caption, assegna all’oggetto un’etichetta per i riferimenti incrociati
% ------------------------------------------------------------------------ %
Il modo migliore per introdurre un oggetto mobile nel sorgente è scriverne il relativo ambiente preceduto e seguito da una riga vuota.
%------------------------------------------------
\dots qui finisce un capoverso.

\begin{table}
\caption{h. . .i}
\label{tab:esempio}
\centering
\begin{tabular}{h. . .i}
...
\end{tabular}
\end{table}

La tabella~\ref{tab:esempio} è un esempio di tabella mobile.
% ------------------------------------------------------------------------ %
\begin{tabular}{lcr}
\begin{array}{cc}
% ------------------------------------------------------------------------ %
Colonne di larghezza prefissata

Il codice seguente, che mostra all’opera il descrittore p{larghezza}, produce la tabella 34:
\begin{tabular}{lp{0.5\textwidth}}
\toprule
\textbf{Forza} & Una forza è una grandezza fisica che si manifesta
nell’interazione di due o più corpi materiali, che cambia lo stato
di quiete o di moto dei corpi stessi. \\
\midrule
\textbf{Momento polare} & Il momento polare di una forza rispetto a
una determinata origine è definito come il prodotto vettoriale tra
il vettore posizione (rispetto alla stessa origine) e la forza. \\
\bottomrule
\end{tabular}
% ------------------------------------------------------------------------ %
nuovo tipo di colonna S specifica per dati numerici 
dell’eventuale testo in una colonna S (di solito nell’intestazione) si scrive tra parentesi graffe per non “confondere” il pacchetto, che lì si aspetterebbe dei numeri

\begin{tabular}%
   {S[table-format=3.1]%
    S[table-format=1.6]}

%
% ------------------------------------------------------------------------ %
% 	IMMAGINI
% ------------------------------------------------------------------------ %
%Principali chiavi di graphicx. 
width Larghezza
height Altezza
scale Larghezza e altezza
angle Orientamento
% ------------------------------------------------------------------------ %
QUAD PER DISTANZIARE I SUBFIG
%
\subfloat[][\emph{Mano con sfera riflettente}.]
   {\includegraphics[width=.45\textwidth]{Sfera}} \quad
nel primo argomento facoltativo di \subfloat, se usato, si mette la didascalia breve da mandare nel relativo indice (\listoffigures o \listoftables )
%
per riferirsi a un sottooggetto in particolare da altre parti del documento, \label va dato dentro il secondo argomento facoltativo imme- diatamente dopo la sottodidascalia
%
% 
% ------------------------------------------------------------------------ %
% 	FIGURE
% ------------------------------------------------------------------------ %
%
\begin{figure}
%
\centering
%
\includegraphics[width=.45\textwidth]{noimage}
%
\caption{image}
%
\label{fig:figure}
%
\end{figure}
%
% ------------------------------------------------------------------------ %
%
%
% 
% ------------------------------------------------------------------------ %
% 	SUB-FIGURE
% ------------------------------------------------------------------------ %
%
\begin{figure}
%
\centering
%
\subfloat[][subcaption]
   {\includegraphics[width=.45\textwidth]{noimage}} \quad
%
\subfloat[][subcaption]
   {\includegraphics[width=.45\textwidth]{noimage}}
%
\caption{image}
%
\label{fig:subfig}
%
\end{figure}
%
% ------------------------------------------------------------------------ %
%
%
% ------------------------------------------------------------------------ %
% 	LISTING
% ------------------------------------------------------------------------ %
%
\lstinputlisting[opzioni]{nome del file con l’estensione}
%
\dots qui finisce un capoverso.
\begin{lstlisting}[caption=%
 {Un codice in display.},label=%
 lst:display]
for i:=maxint to 0 do
begin
   { non far nulla }
end;
write(’Benvenuto in Pascal.’);
\end{lstlisting}
% ------------------------------------------------------------------------ %
Si può interrompere un codice e riprenderlo successivamente mantenendo la numerazione corretta
% ------------------------------------------------------------------------ %
Scrivendo nel preambolo (si noti l’argomento vuoto):
\lstnewenvironment{pascal}{\lstset{language=pascal}}{} si potrà poi usare l’ambiente pascal nel modo seguente:
% ------------------------------------------------------------------------ %