% ------------------------------------------------------------------------ %
% !TEX encoding = UTF-8 Unicode
% !TEX TS-program = pdflatex
% !TEX root = ../Tesi.tex
% !TEX spellcheck = it-IT
% ------------------------------------------------------------------------ %
%
% ------------------------------------------------------------------------ %
% 	ACRONIMI
% ------------------------------------------------------------------------ %
%
\cleardoublepage
%
\chapter{Acronyms}
%
\markboth{Acronyms}{Acronyms}	% headings
%
\begin{acronym}[OpenFOAM]	% tra [ ] inserire l'acronimo più lungo
%
% ------------------------------------------------------------------------ %
%
% tra [ ] inserire come deve apparire l'acronimo nel testo
%
% ------------------------------------------------------------------------ %
%
\begin{otherlanguage*}{english}
%
\acro{BMS}[BMS]{Building Management System}

{\smaller A building management system is a computer-based control system installed in buildings that controls and monitors the building’s mechanical and electrical equipment such as ventilation, lighting, power systems, fire systems, and security systems.\\
\href{https://en.wikipedia.org/wiki/Building_management_system}{www.en.wikipedia.org}
\par}
%
\end{otherlanguage*}
%
% ------------------------------------------------------------------------ %
%
\acro{HVAC}[HVAC]{Heating, Ventilation and Air Conditioning}

{\smaller Heating, ventilation and air conditioning is the technology of indoor and vehicular environmental comfort. Its goal is to provide thermal comfort and acceptable indoor air quality. HVAC is an important part of residential structures such as single family homes, apartment buildings, hotels, medium to large industrial and office buildings where safe and healthy building conditions are regulated with respect to temperature and humidity, using fresh air from outdoors.\\
\href{https://en.wikipedia.org/wiki/HVAC}{www.en.wikipedia.org}
\par}
%
% ------------------------------------------------------------------------ %
%
\begin{otherlanguage*}{english}
%
\acro{CAD}[CAD]{Computer-Aided Design}

{\smaller Computer-aided design (CAD) is the use of computer systems to aid in the creation, modification, analysis, or optimization of a design. CAD software is used to increase the productivity of the designer, improve the quality of design, improve communications through documentation, and to create a database for manufacturing. CAD output is often in the form of electronic files for print, machining, or other manufacturing operations.\\
\href{https://en.wikipedia.org/wiki/Computer-aided_design}{www.en.wikipedia.org}
\par}
%
\end{otherlanguage*}
%
% ------------------------------------------------------------------------ %
%
\acro{IPS}[IPS]{Indoor Positioning System}

{\smaller An indoor positioning system is a system to locate objects or people inside a building using radio waves, magnetic fields, acoustic signals, or other sensory information collected by mobile devices. There are several commercial systems on the market, but there is no standard for an IPS system.
IPSes use different technologies, including distance measurement to nearby anchor nodes (nodes with known positions, e.g., WiFi access points), magnetic positioning, dead reckoning. They either actively locate mobile devices and tags or provide ambient location or environmental context for devices to get sensed.\\
\href{https://en.wikipedia.org/wiki/Indoor_positioning_system}{www.en.wikipedia.org}
\par}
%
% ------------------------------------------------------------------------ %
%
%
\acro{BLE}[BLE]{Bluetooth Low Energy}

{\smaller Bluetooth low energy (Bluetooth LE, BLE, marketed as Bluetooth Smart) is a wireless personal area network technology designed and marketed by the Bluetooth Special Interest Group aimed at novel applications in the healthcare, fitness, beacons, security, and home entertainment industries. Compared to Classic Bluetooth, Bluetooth Smart is intended to provide considerably reduced power consumption and cost while maintaining a similar communication range.\\
\href{https://en.wikipedia.org/wiki/Bluetooth_low_energy}{www.en.wikipedia.org}
\par}


\acro{RFID}[RFID]{Radio-Frequency IDentification}

{\smaller Radio-frequency identification uses electromagnetic fields to automatically identify and track tags attached to objects. The tags contain electronically stored information. Passive tags collect energy from a nearby RFID reader's interrogating radio waves. Active tags have a local power source such as a battery and may operate at hundreds of meters from the RFID reader. Unlike a barcode, the tag need not be within the line of sight of the reader, so it may be embedded in the tracked object. RFID is one method for Automatic Identification and Data Capture (AIDC).\\
\href{https://en.wikipedia.org/wiki/Radio-frequency_identification}{www.en.wikipedia.org}
\par}


\acro{PIR}[PIR]{Passive InfraRed sensor}

{\smaller A passive infrared sensor (PIR sensor) is an electronic sensor that measures infrared (IR) light radiating from objects in its field of view. They are most often used in PIR-based motion detectors.\\
\href{https://en.wikipedia.org/wiki/Passive_infrared_sensor}{www.en.wikipedia.org}
\par}

\acro{RSS}[RSS]{Received Signal Strength}

{\smaller In telecommunications, received signal strength, or received signal strength indicator (RSSI) is a measurement of the power present in a received radio signal. IEEE 802.11 devices often make the measurement available to users.
RSS is often done in the intermediate frequency (IF) stage before the IF amplifier. In zero-IF systems, it is done in the baseband signal chain, before the baseband amplifier. RSSI output is often a DC analog level. It can also be sampled by an internal ADC and the resulting codes available directly or via peripheral or internal processor bus.\\
\href{https://en.wikipedia.org/wiki/Received_signal_strength_indication}{www.en.wikipedia.org}
\par}

% ------------------------------------------------------------------------ %
\end{acronym}
%
% ------------------------------------------------------------------------ %