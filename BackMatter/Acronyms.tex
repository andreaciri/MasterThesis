% ------------------------------------------------------------------------ %
% !TEX encoding = UTF-8 Unicode
% !TEX TS-program = pdflatex
% !TEX root = ../Tesi.tex
% !TEX spellcheck = it-IT
% ------------------------------------------------------------------------ %
%
% ------------------------------------------------------------------------ %
% 	ACRONIMI
% ------------------------------------------------------------------------ %
%
\cleardoublepage
%
\chapter{Acronyms}
%
\markboth{Acronyms}{Acronyms}	% headings
%
\begin{acronym}[OpenFOAM]	% tra [ ] inserire l'acronimo più lungo
%
% ------------------------------------------------------------------------ %
%
% tra [ ] inserire come deve apparire l'acronimo nel testo
%
% ------------------------------------------------------------------------ %
%
\begin{otherlanguage*}{english}
%
\acro{BMS}[BMS]{Building Management System}

{\smaller A building management system is a computer-based control system installed in buildings that controls and monitors the building’s mechanical and electrical equipment such as ventilation, lighting, power systems, fire systems, and security systems.\\
\href{https://en.wikipedia.org/wiki/Building_management_system}{www.en.wikipedia.org}
\par}
%
\end{otherlanguage*}
%
% ------------------------------------------------------------------------ %
%
\acro{HVAC}[HVAC]{Heating, Ventilation and Air Conditioning}

{\smaller Heating, ventilation and air conditioning is the technology of indoor and vehicular environmental comfort. Its goal is to provide thermal comfort and acceptable indoor air quality. HVAC is an important part of buildings where safe and healthy building conditions are regulated with respect to temperature and humidity, using fresh air from outdoors.\\
\href{https://en.wikipedia.org/wiki/HVAC}{www.en.wikipedia.org}
\par}
%
% ------------------------------------------------------------------------ %
%
\begin{otherlanguage*}{english}
%
\acro{CAD}[CAD]{Computer-Aided Design}

{\smaller Computer-aided design (CAD) is the use of computer systems to aid in the creation, modification, analysis, or optimization of a design. CAD software is used to increase the productivity of the designer and improve the quality of design.\\
\href{https://en.wikipedia.org/wiki/Computer-aided_design}{www.en.wikipedia.org}
\par}
%
\end{otherlanguage*}
%
% ------------------------------------------------------------------------ %
%
\acro{IPS}[IPS]{Indoor Positioning System}

{\smaller An indoor positioning system locates objects or people inside a building using radio waves, magnetic fields, acoustic signals, or other information collected by mobile devices.
IPSes use different technologies, including distance measurement to nearby anchor nodes (nodes with known positions, e.g., WiFi access points), magnetic positioning, dead reckoning.\\
\href{https://en.wikipedia.org/wiki/Indoor_positioning_system}{www.en.wikipedia.org}
\par}
%
% ------------------------------------------------------------------------ %
%
%
\acro{BLE}[BLE]{Bluetooth Low Energy}

{\smaller Bluetooth low energy (Bluetooth LE, BLE, or Bluetooth Smart) is a wireless personal area network technology aimed at novel applications in the healthcare, beacons, security, and home entertainment industries. Compared to Classic Bluetooth, BLE is intended to provide considerably reduced power consumption and cost while maintaining a similar communication range.\\
\href{https://en.wikipedia.org/wiki/Bluetooth_low_energy}{www.en.wikipedia.org}
\par}


\acro{RFID}[RFID]{Radio-Frequency IDentification}

{\smaller Radio-frequency identification uses electromagnetic fields to automatically identify and track tags attached to objects. The tags contain electronically stored information. Passive tags collect energy from a nearby RFID reader's interrogating radio waves. Active tags have a local power source and may operate at hundreds of meters from the reader.\\
\href{https://en.wikipedia.org/wiki/Radio-frequency_identification}{www.en.wikipedia.org}
\par}


\acro{PIR}[PIR]{Passive InfraRed sensor}

{\smaller A passive infrared sensor (PIR sensor) is an electronic sensor that measures infrared (IR) light radiating from objects in its field of view. They are most often used in PIR-based motion detectors.\\
\href{https://en.wikipedia.org/wiki/Passive_infrared_sensor}{www.en.wikipedia.org}
\par}

\acro{RSS}[RSS]{Received Signal Strength}

{\smaller Received signal strength is a measurement of the power present in a received radio signal. RSS output is often a DC analog level. It can also be sampled by an internal ADC and the resulting codes available directly or via peripheral or internal processor bus.\\
\href{https://en.wikipedia.org/wiki/Received_signal_strength_indication}{www.en.wikipedia.org}
\par}

\acro{UUID}[UUID]{Universally Unique IDentifier}

{\smaller A universally unique identifier is an identifier standard used in software construction. A UUID is simply a 128-bit value. The meaning of each bit is defined by any of several variants.\\
\href{https://en.wikipedia.org/wiki/Universally_unique_identifier}{www.en.wikipedia.org}
\par}

\acro{KNN}[KNN]{K-Nearest Neighbor}

{\smaller K-Nearest Neighbors algorithm is a non-parametric method used for classification and regression. In both cases, the input consists of the k closest training examples in the feature space.\\
\href{https://en.wikipedia.org/wiki/K-nearest_neighbors_algorithm}{www.en.wikipedia.org}
\par}

\acro{VNS}[VNS]{Variable Neighborhood Search}

{\smaller Variable neighborhood search is a metaheuristic method for solving a set of combinatorial optimization and global optimization problems. It explores distant neighborhoods of the current incumbent solution, and moves from there to a new one if and only if an improvement was made.\\
\href{https://en.wikipedia.org/wiki/Variable_neighborhood_search}{www.en.wikipedia.org}
\par}

% ------------------------------------------------------------------------ %
\end{acronym}
%
% ------------------------------------------------------------------------ %