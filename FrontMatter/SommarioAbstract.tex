% ------------------------------------------------------------------------ %
% !TEX encoding = UTF-8 Unicode
% !TEX TS-program = pdflatex
% !TEX root = ../Tesi.tex
% !TEX spellcheck = it-IT
% ------------------------------------------------------------------------ %
%
% ------------------------------------------------------------------------ %
% 	SOMMARIO + ABSTRACT
% ------------------------------------------------------------------------ %
%
\cleardoublepage
%
\phantomsection
%
\pdfbookmark{Abstract}{Abstract}
%
\markboth{Abstract}{Abstract}
% ------------------------------------------------------------------------ %
% consento presenza di più capitoli nella stessa pagina
\begingroup
%\let\clearpage\relax
\let\cleardoublepage\relax
\let\cleardoublepage\relax
% ------------------------------------------------------------------------ %
%
\chapter*{Abstract}
%
During the past few years, a lot of effort has been put in the research area of connected sensors and actuators, the so-called \emph{Internet-Of-Things}. A big portion of this research is applied to the domain of Smart Buildings, where connected devices are exploited to increase comfort, safety, and energy sustainability.
However, there can be identified some missing tiles that get more difficult the adoption of such technologies in real buildings.
The work done in this thesis tries to solve at least two of these missing tiles.

First, is analyzed the problem of knowing number and position of building occupants, and an \emph{Occupancy Monitoring} system is proposed. One of the main limitations of existing solutions is the inability to provide both long time monitoring (many hours of execution) and a non-intrusive approach from the occupant point of view.
With the proposed system, called BlueSentinel, we tried to exploit the Bluetooth Low Energy technology at low level to detect the presence of occupants inside the building. The approach allows the system to run continuously and for many hours without heavily affect the mobile battery consumption.\\
The experimental results confirmed that the proposed approach is suitable to obtain a real-time occupancy information with a localization accuracy comparable to the state of the art.

The second part of this work tries to solve the problem of designing optimal Smart Building systems based on wireless devices. We tried to explain the challenges faced by designers during the installation of Smart Building systems that require the positioning of several hardware nodes, and the lack of automatic tools able to support this phase. A common limitation of the few existing models is the lack of a convenient way to specify geometric information of the indoor map.\\
For these reasons, we propose an integrated CAD tool for both floor plan specification and node positioning, developed within an \mbox{open-source} CAD environment extensible through plug-ins. The tool is able to provide a near-optimal solution of node allocations, possibly with mixed types, with the aim to reduce the installation costs. The results suggest that, depending on the available hardware types, total cost of the solution could be improved by the proposed model moving from homogeneous to mixed type allocation.

\medskip
%
\noindent \textbf{Keywords:} 
Smart Buildings; Occupancy Detection; Occupancy Monitoring; Indoor Localization; Indoor WSNs; Indoor WSNs Deployment;
%
\clearpage
%\vfill
%
% ------------------------------------------------------------------------ %
%
%
\pdfbookmark{Sommario}{Sommario}
%
\markboth{Sommario}{Sommario}
%
\chapter*{Sommario}
%
Negli ultimi anni, il continuo avanzamento tecnologico nel campo delle reti di sensori e attuatori ha generato il fenomeno del cosiddetto \emph{Internet-Of-Things}. Larga parte della ricerca in questo campo è rivolto alla realizzazione di \emph{edifici intelligenti} (Smart Buildings), in cui numerosi dispositivi connessi sono utilizzati per aumentare il comfort, la sicurezza ed il risparmio energetico degli edifici. Tuttavia, esistono ancora alcuni problemi irrisolti che ostacolano la diffusione di queste tecnologie negli edifici odierni. Il lavoro descritto in questa tesi tenta di risolvere almeno due di questi problemi.

Per prima cosa viene affrontato il problema di rilevare in tempo reale il numero e la posizione degli utenti all'interno di un edificio. I sistemi esistenti in letteratura non raggiungono lunghi tempi di esecuzione, o utilizzano approcci invasivi per gli abitanti dell'ambiente.
A tal proposito verrà presentato BlueSentinel, il sistema di \emph{Occupancy Monitoring} realizzato sfruttando il protocollo Bluetooth Low Energy per monitorare a lungo la posizione degli utenti, senza gravare sul consumo di batteria mobile.\\
Le analisi sperimentali confermano che l'approccio proposto è adatto a rilevare in tempo reale la posizione degli utenti, con un'accuratezza della localizzazione in linea con i metodi proposti nello stato dell'arte.

Successivamente, viene affrontato il problema del design di sistemi per Smart Buildings basati su nodi wireless. Durante l'installazione di questi sistemi, ad esempio di localizzazione o monitoraggio, il posizionamento dei nodi è critico, influendo sul costo e le performance del sistema. Tuttavia, questo processo viene ancora realizzato manualmente a causa della mancanza di tool automatizzati.\\
Proponiamo dunque una applicazione CAD per la specifica della struttura di un edificio e l'allocamento ottimale di nodi sensori. L'applicazione, rilasciata in maniera open-source ed estendibile tramite plug-ins, è in grado di posizionare i nodi in modo da massimizzare l'accuratezza del sistema, e allo stesso tempo fornire soluzioni a basso costo.
I risultati mostrano che il tool proposto è in grado di fornire, a parità di accuratezza, soluzioni meno costose allocando nodi eterogenei.

\medskip
%
\noindent \textbf{Parole chiave:} 
Edifici Smart; Rilevamento dell'Occupazione; Monitoraggio dell'Occupazione; Localizzazione Indoor; Indoor WSNs; Posizionamento Automatico di WSNs Indoor;
%
% ------------------------------------------------------------------------ %
%
%
\endgroup			
%
%\vfill
%
% ------------------------------------------------------------------------ %